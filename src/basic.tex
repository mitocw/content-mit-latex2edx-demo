%%%%%%%%%%%%%%%%%%%%%%%%%%%%%%%%%%%%%%%%

\begin{edXchapter}{Basic examples}

%%%%%%%%%%%%%%%%%%%%%%%%%%%%%%%%%%%%%%%%%%%%%%%%%%%%%%%%%%%%%%%%%%%%%%%%%%%%%

\begin{edXsection}{Basic example problems}

\begin{edXvertical}

\begin{edXproblem}{Option response}

This is a sample problem, which is worth 10 points.

Give the correct python {\tt type} for the following expressions.  Select {\tt noneType} if the expression is illegal.

\begin{itemize}
\item \edXinline{\tt 3~~~}   \edXabox{expect="int" options="noneType","int","float" type="option" inline="1"}
\item \edXinline{\tt 5.2~~~} \edXabox{expect="float" options="noneType","int","float" type="option" inline="1"}
\item \edXinline{\tt 3/2~~~} \edXabox{expect="int" options="noneType","int","float" type="option" inline="1"}
\item \edXinline{\tt 1+[]~~~} \edXabox{expect="noneType" options="noneType","int","float" type="option" inline="1"}
\end{itemize}

\end{edXproblem}

%%%%%%%%%%%%%%%%%%%%%%%%%%%%%%%%%%%%%%%%

\begin{edXproblem}{String response}

What state is Detroit in?

\edXabox{expect="Michigan" type="string"}

\begin{edXsolution}

Explanations can also be provided inside:
\begin{verbatim}
\begin{edXsolution}
... 
\end{edXsolution}
\end{verbatim}

\end{edXsolution}


\end{edXproblem}

%%%%%%%%%%%%%%%%%%%%%%%%%%%%%%%%%%%%%%%%

\begin{edXproblem}{Numerical response}

\section{Example of numerical response}  

What is the numerical value of $pi$?

\edXabox{expect="3.14159" type="numerical" tolerance='0.01' }

\end{edXproblem}

%%%%%%%%%%%%%%%%%%%%%%%%%%%%%%%%%%%%%%%%

\begin{edXproblem}{Custom response}

\section{Example of custom response}  

This problem demonstrates the use of a custom python script used for
checking the answer.

\begin{edXscript}

def sumtest(expect,ans):
    try:
        (a1,a2) = map(float,ans)
        return (a1+a2)==10
    except Exception as err:
        return {'ok': False, 'msg': 'Sorry, cannot evaluate your input ' + str(ans)}

\end{edXscript}

Enter two numbers which add up to 10:

\edXabox{expect=""
  type="custom"
  answers="1,9"
  prompts="x = ","y = "
  cfn="sumtest"
  inline="1" }%

\end{edXproblem}

\end{edXvertical}

\end{edXsection}

%%%%%%%%%%%%%%%%%%%%%%%%%%%%%%%%%%%%%%%%%%%%%%%%%%%%%%%%%%%%%%%%%%%%%%%%%%%%%

\begin{edXsection}{Video and text}

\begin{edXvertical}

\begin{edXtext}{Sample show-hide text section}

{\LARGE Sample show-hide text section}

Pieces of text (and images) can be put inside a ``showhide'' section, which the user can hide or show via a click.

\begin{edXshowhide}{ps4starkket}{Hints and instructions for entering expressions}

Note that rotations leave points on the axis of rotation unmoved.  For
a point $(\theta, \phi)$ specified in polar coordinates on the surface
of the \href{http://en.wikipedia.org/wiki/Bloch_sphere}{Bloch sphere},
the corresponding two-level quantum state is
\bea
	|\psi\> = \cos\frac{\theta}{2} |a\> +
        e^{i\phi}\sin\frac{\theta}{2} |b\>
\,.
\eea

Enter the state using the vertical bar ${\tt |}$ and greater-than
symbol ${\tt >}$ to delineate a ``ket'' and enter $\omega_0$ as ${\tt omega\_0}$ as usual.

Expressions like ${\tt (2*|a> + |b>)/sqrt(3)}$ are legal input
(remember to include ${\tt *}$ to denote multiplication, for
coefficients in front of kets).

Standard mathematical functions may be employed, e.g. ${\tt sin(x)}$,
${\tt sqrt(x)}$, ${\tt arccos(x)}$,  ${\tt arctan(x)}$, etc.  

\end{edXshowhide}

\end{edXvertical}

\begin{edXvertical}

\edXxml{<video url_name="week0_intro" display_name="Introduction to 6.SFMx" youtube="1.0:VfOMvuHcJi0" />}

\end{edXvertical}

\end{edXsection}


%%%%%%%%%%%%%%%%%%%%%%%%%%%%%%%%%%%%%%%%%%%%%%%%%%%%%%%%%%%%%%%%%%%%%%%%%%%%%
\end{edXchapter}
